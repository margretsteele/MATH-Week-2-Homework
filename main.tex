\documentclass{exam}
%\usepackage[utf8]{inputenc}
\usepackage{systeme}
\usepackage{url}
\usepackage{amsmath,amsfonts,amssymb,amsthm}

\title{Homework 2}
\date{April, 24th 2017}


\begin{document}

\maketitle
\section{Subspaces}
\subsection{True or False? Let A be a 5x3 matrix. Then the range of A is a subspace in $\mathbb{R}^3$.}
\begin{oneparchoices}
 \choice True
 \choice False
\end{oneparchoices}

\subsection{Orthogonal complement of a set}
Given a set $S$, its orthogonal complement is defined as $S^{\perp} = \{x: x^Ty = 0 \mbox{ for all } y \in S\}$.
Is $S^{\perp}$ a subspace?
\begin{oneparchoices}
 \choice Yes
 \choice No 
 \choice Depends on $S$
\end{oneparchoices}


\subsection{Null space}
Find the basis vector for null-space of $A = \begin{bmatrix}
            1 & 2 \\
            2 & 4
          \end{bmatrix}$

\begin{oneparchoices}
\choice $\begin{bmatrix} 2 \\ 1 \end{bmatrix}$
\choice $\begin{bmatrix} 2 \\ -1 \end{bmatrix}$
\choice $\begin{bmatrix} -2 \\ -2 \end{bmatrix}$
\choice Null-space is the $0$ vector

\end{oneparchoices}
       


\section{Matrix representation}
Let $f : \mathbb{R}^2 \to \mathbb{R}^2$ be a linear operator, and $f((1, 0)) = (1, 3), f((0, 1)) = (2, −1)$. What is the matrix representation of f under the ordered bases ${(1, 0), (0, 1)}$?

\begin{oneparchoices}
  \choice $\begin{bmatrix}
            1 & 3 \\
            2 & -1
          \end{bmatrix}$

  \choice $\begin{bmatrix}
            1 & 2 \\
            3 & -1
          \end{bmatrix}$

  \choice $\begin{bmatrix}
            -1 & 3 \\
            2 & 1
          \end{bmatrix}$

  \choice $\begin{bmatrix}
            2 & 1 \\
            -1 & 3
          \end{bmatrix}$
\end{oneparchoices}

\section{Linear transform I}
Let $T(x) = (3x_1, 4\sqrt{x_1}*\sqrt{x_2}, x_1 + x_2)$.

\begin{oneparchoices}
\choice $T$ is a linear transform
\choice $T$ is not a linear transform and violates additive property
\choice $T$ is not a linear transform and violates multiplicative property
\end{oneparchoices}

\section{Linear transform II}
Let $T(x) = (3x_1, 0, x_1 - x_2)$.

\begin{oneparchoices}
\choice $T$ is a linear transform
\choice $T$ is not a linear transform and violates additive property
\choice $T$ is not a linear transform and violates multiplicative property
\end{oneparchoices}

\section{Eigen-values}
\[
  A=
\left[ {\begin{array}{cc}
   -7 & -4 \\
   33 & 16
  \end{array} } \right]
\]
The eigen-values of $A$ are:
\begin{oneparchoices}
 \choice -4  
 \choice 4,5
 \choice 5
 \choice -4,-5
\end{oneparchoices}

\section{Eigen-vectors}
For the given matrix and eigenvalue, find an eigenvector corresponding to the eigenvalue.
\[
  A=
\left[ {\begin{array}{cc}
   -10 & -1 \\
   56 & 5
  \end{array} } \right]
\]

\begin{center}
\(\lambda\) = -2
\end{center}

\begin{oneparchoices}

  \choice $\begin{bmatrix}
            1 \\
            5
          \end{bmatrix}$
  \choice $\begin{bmatrix}
            -8 \\
            1
          \end{bmatrix}$

  \choice $\begin{bmatrix}
            1 \\
            0
          \end{bmatrix}$

  \choice $\begin{bmatrix}
            1 \\
            -8
          \end{bmatrix}$
\end{oneparchoices}

\section{Eigen-space}
For the given matrix A, find a basis for the corresponding eigen-space for the given eigenvalue.
\[
  A=
\left[ {\begin{array}{ccc}
   4 & 0 & 0 \\
   27 & -5 & 0 \\
   102 & -28 & 2
  \end{array} } \right]
\]

\begin{center}
\(\lambda\) = 4
\end{center}

\begin{oneparchoices}
  \choice $\begin{Bmatrix}
          \begin{bmatrix}
            1 \\
            3 \\
            0
          \end{bmatrix},
          \begin{bmatrix}
            1 \\
            0 \\
            9
          \end{bmatrix}
          \end{Bmatrix}$

  \choice $\begin{Bmatrix}
          \begin{bmatrix}
            1 \\
            3 \\
            0
          \end{bmatrix},
          \begin{bmatrix}
            1 \\
            0 \\
            -9
          \end{bmatrix}
          \end{Bmatrix}$

  \choice $\begin{Bmatrix} 
          \begin{bmatrix}
            1 \\
            3 \\ 
            9
          \end{bmatrix}
          \end{Bmatrix}$

  \choice $\begin{Bmatrix} 
          \begin{bmatrix}
            1 \\
            -3 \\
            -9
          \end{bmatrix}
          \end{Bmatrix}$
\end{oneparchoices}
\vspace{5mm}
\\
\textbf{Fun fact:} Eigen-faces is an interesting application of eigen-spaces in machine learning/computer-vision that identifies fundamental face-types
given a set of images of people's faces.

\section{SVD and eigen-values}
Let
$$
  A=
\left[ {\begin{array}{ccc}
   3 & 2 & 2 \\
   2 & 3 & -2
  \end{array} } \right]
$$

Let the eigen-vectors of $AA^T$ be:
$$
X = 
\left[ \begin{array}{cc}
 0.71 & -0.71 \\
  0.71  & 0.71
\end{array} \right]
$$
with eigen-values $25, 9$.

Let the eigen-vectors of $A^TA$ be:
$$
Y = 
\left[\begin{array}{ccc}
-0.71 & -0.67 & 0.24 \\
-0.71 & 0.67 & -0.24 \\
0 & 0.33 & 0.94 
\end{array} \right]
$$
with eigen-values $25, 0, 9$.


\subsection{Left singular vectors of $A$}
The left singular vectors of $A$ are given by:
\begin{oneparchoices}
 \choice $XY$
 \choice $X$
 \choice $XY^T$
 \choice $Y$
\end{oneparchoices}

\subsection{Right singular vectors of $A$}
The right singular vectors of $A$ are given by:
\begin{oneparchoices}
 \choice $XY$
 \choice $X$
 \choice $XY^T$
 \choice $Y$
\end{oneparchoices}

\subsection{Singular values of $A$}
The singular values of $A$ are given by:
\begin{oneparchoices}
\choice 25, 9
\choice 5, 3
\choice 25, 9, 0
\choice 5, 3, 0
\end{oneparchoices}

\subsection{Eigen-values of $A$}
The eigen values of $A$ are given by:
\begin{oneparchoices}
\choice 25, 9
\choice 5, 3, 0
\choice 5, 3
\choice Not defined
\end{oneparchoices}

\subsection{Raising to the power n}
Consider $B = (AA^T)^n$ i.e. the matrix $AA^T$ multiplied to itself $n$ times.
The eigen-vectors of $B$ are: \\
\begin{oneparchoices}
\choice $X^n$
\choice $X$
\choice $X^{2n}$
\choice None of the above
\end{oneparchoices}
\\
\vspace{5mm}
\\
\textbf{Fun fact:} Raising a square symmetric matrix, $C$ to the power $n$ shows up in Markov chains, an important topic in machine learning,
where the matrix $C$ is the state transition probability matrix.

\section{Recommender systems}
Let's say you are tasked with building a recommender system for Amazon.
You are given a matrix $M \in \mathbb{R}^{m \times n}$ where the rows corresponds to customers
and columns corresponding to asins. When a customer purchases an asin, he gives
a rating between 1 and 5 to the asin (higher the better). Unfortunately, the matrix
is in-complete as customers only purchase a small fraction of the products Amazon sells.
\subsection{A simple recommender}
What would be a good baseline recommender that completes (or fills in) this incomplete matrix,
so that you can recommend asins with high ratings to users?

\begin{choices}
 \choice Fill in zeros if a customer hasn't purchased a product
 
  \choice For each customer x, and for each product y the customer hasn't purchased, fill in the average rating of other customers who have purchased the product y.
  
 \choice For each customer x, and for each product y the customer x hasn't purchased, fill in the average rating of the products
 the customer x has purchased.
 
\end{choices}

\subsection{Storing the recommender}
You decide to store the recommendations from your base-line recommender but observe that you run out of memory.
What would be a more efficient yet accurate storage solution (i.e. the storage should be $ << O(m*n)$)

\begin{choices}
 \choice Compute determinants of all the sub-matrices of the recommendation matrix and store them and reconstruct the recommendations at prediction time.
 
 \choice Store the Eigen values and the diagonals of the matrix and reconstruct the recommendations at prediction time.
 
 \choice Do a partial SVD of the recommendation matrix and store the relevant factors and reconstruct recommendations at prediction time.
 
 \choice Compute a complete SVD  of the recommendation matrix and store the factors and reconstruct recommendations at prediction time.
\end{choices}

\subsection{Making predictions}
Given your storage solution from the previous question, what is the computational cost of predicting a single customers ratings for products:

\begin{oneparchoices}
\choice $O(n)$
\choice $O(m)$
\choice $O(m\log(n))$
\choice $O(n\log(m))$
\end{oneparchoices}


\section{Bonus: Determine a representation for a linear transformation}
Let 
\begin{eqnarray}
\left[ \begin{array}{c} 3 \\ 1 \end{array} \right],
\left[ \begin{array}{c} 1 \\ 2 \end{array} \right],
\end{eqnarray}
be two eigenvectors for a transformation $T$ with eigenvalues 2 and 1 respectively. So, with respect to the two eigenvectors $T$ can be written as
\[
  T =
\left( {\begin{array}{cccc}
   2 & 0 \\
   0 & 1
  \end{array} } \right)
\]
Find the representation of $T$ (call it $T'$) with respect to the canonical basis vectors
\begin{eqnarray}
\left[ \begin{array}{c} 1 \\ 0 \end{array} \right],
\left[ \begin{array}{c} 0 \\ 1 \end{array} \right],
\end{eqnarray}

Hint: One solution is to find a matrix $Q$ such that $$T' = Q^{-1} \cdot T \cdot Q$$ where $Q$ represents the change in base. Make sure that when $T'$ is applied to the two eigenvectors you get the desired result. \\

\begin{oneparchoices}
  \choice $\begin{bmatrix}
            2 & 0 \\
            0 & 1
          \end{bmatrix}$

  \choice $\begin{bmatrix}
            11/5 & -3/5 \\
            2/5 & 4/5
          \end{bmatrix}$

  \choice $\begin{bmatrix}
            2/5 & -1/5 \\
            1/5 & 3/5
          \end{bmatrix}$

  \choice $\begin{bmatrix}
            3 & 1 \\
            1 & 2
          \end{bmatrix}$
\end{oneparchoices}


\end{document}